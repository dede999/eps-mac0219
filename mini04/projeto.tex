\documentclass[12pt]{article}
\usepackage[margin=2cm]{geometry}
\usepackage[utf8]{inputenc}
\usepackage{indentfirst}
\usepackage{hyperref}
\usepackage{amsmath}
\usepackage[brazil]{babel}

\title{MiniEp 04 -- Overhead e Starvation em algoritmos de acesso à SC}
\author{André Luiz Abdalla Silveira -- $8030353$}

\begin{document}

\maketitle
\section*{Introdução}

Nesse EP, somos convocados à contemplar e refletir acerca dos algoritmos apresentados nos arquivos \texttt{bakery.c} e \texttt{gate.c}, bem como fazer alterações para verificar e apontar determinados efeitos ou fenômenos. A primeira parte desse trabalho consiste em apresentar os dados obtidos, e a segunda. 

É possível notar que a presença do \_\_sync\_syncronize (ss) permite que os testes em \texttt{test.c} sejam bem sucedidos ao invés do que acontece quando essa instrução não está presente. Entretanto, esse é o único efeito observável.

Além de testar os resultados com e sem \_\_sync\_syncronize, outras intervenções efetuadas envolvem a escolha dos escalonadores, e quaisquer alterações que se fizeram necessárias para a devida execução desses algoritmos, como comentar um \texttt{return 1} na linha 109 do arquivo de testes. Os efeitos dessas alterações encontram-se comentados abaixo

\section*{Legenda}
\begin{itemize}
\item $T_{alg}$ é a média obtida nos "Elapsed Time" para as 30 execuções do algoritmo \textit{alg}\footnote{$\mathit{alg} \in \{bakery, gate\}$}
\item $\mu_{alg}$ média obtida considerando-se as quantidades médias de acesso à SC para \textit{alg}
\item $\sigma_{alg}$ média obtida considerando-se os desvios padrão relativos às quantidades médias de acesso à SC para \textit{alg}
\item \texttt{\textbf{SCHED\_OTHER}} é o escalonador padrão do sistema operacional
\item \texttt{\textbf{SCHED\_FIFO}} é o escalonador \texttt{First In, First Out}
\item \texttt{\textbf{SCHED\_RR}} é o escalonador \texttt{Round Robbin}
\end{itemize}
 \newpage

\section{Tabelas}

\begin{center}
	\begin{table}[h!]
      \begin{tabular}{|cc|ccc|ccc|}
          \hline
            Nº Threads & Tempo total & $T_{bakery}$ & $\mu_{bakery}$ & $\sigma_{bakery}$ & $T_{gate}$ & $\mu_{gate}$ & $\sigma_{gate}$ \\ \hline \hline
            20 & 3000000 & 230.69 & 1500.00 & 143.47 & 3.39 & 1872.87 & 3252.87 \\
            30 & 3000000 & 388.21 & 1000.00 & 34.00 & 3.67 & 1251.83 & 2641.13  \\ \hline
            60 & 3000000 & 757.86 & 500.00 & 8.97 & 3.79 & 593.20 & 1954.27  \\ \hline
            60 & 6000000 & 1201.03 & 1000.00 & 10.20 & 27.97 & 1153.87 & 3318.77  \\
            60 & 9000000 & 1788.04 & 1500.00 & 11.10 & 47.04 & 1692.90 & 4309.57  \\ \hline
    	\end{tabular}
        \caption{Resultados obtidos sem \_\_sync\_syncronize}
        \label{table:s-ss}
	\end{table}

	\begin{table}[h!]
    	\begin{tabular}{|cc|ccc|ccc|}
          \hline
            Nº Threads & Tempo total & $T_{bakery}$ & $\mu_{bakery}$ & $\sigma_{bakery}$ & $T_{gate}$ & $\mu_{gate}$ & $\sigma_{gate}$ \\ \hline \hline
            20 & 3000000 & 238.69 & 1500.03 & 142.67 & 3.20 & 1802.07 & 3155.90 \\
            30 & 3000000 & 366.49 & 1000.00 & 46.70 & 3.53 & 1246.47 & 2790.07  \\ \hline
            60 & 3000000 & 706.51 & 500.00 & 12.87 & 4.89 & 579.93 & 1878.17 \\ \hline
            60 & 6000000 & 1386.40 & 1000.00 & 9.70 & 35.04 & 1253.13 & 3575.74  \\
            60 & 9000000 & 1939.28 & 1500.00 & 9.63 & 80.37 & 1687.27 & 4481.27  \\ \hline
		\end{tabular}
        \caption{Resultados obtidos com \_\_sync\_syncronize}
        \label{table:c-ss}
	\end{table}
    
	\begin{table}[h]
      \begin{tabular}{|cc|ccc|ccc|}
          \hline
            Escalonador  & ss? & $T_{bakery}$ & $\mu_{bakery}$ & $\sigma_{bakery}$ & $T_{gate}$ & $\mu_{gate}$ & $\sigma_{gate}$ \\ \hline \hline
            \texttt{\textbf{SCHED\_OTHER}}& \textbf{N} & 388.21 & 1000.00 & 34.00 & 3.67 & 1251.83 & 2641.13  \\
            \texttt{\textbf{SCHED\_OTHER}}& \textbf{S} & 366.49 & 1000.00 & 46.70 & 3.53 & 1246.47 & 2790.07  \\ \hline \hline
             \texttt{\textbf{SCHED\_FIFO}}& \textbf{N} & 50.50 & 1000.00 & 3.53 & 1.11 & 1278.67 & 2351.63  \\
            \texttt{\textbf{SCHED\_FIFO}}& \textbf{S} & 47.17 & 1000.00 & 2.57 & 0.14 & 1000.00 & 1413.53  \\ \hline \hline
             \texttt{\textbf{SCHED\_RR}}& \textbf{N} & 38.53 & 1000.00 & 4.08 & 0.96 & 1325.97 & 2365.50  \\
            \texttt{\textbf{SCHED\_RR}}& \textbf{S} & 39.72 & 1000.00 & 2.80 & 1.60 & 1000.00 & 1427.13  \\ \hline \hline
    	\end{tabular}
        \caption{Resultados obtidos com \_\_sync\_syncronize, 30 threads, 3 M de tempo total}
        \label{escalador}
	\end{table}
\end{center}

\section{Conclusões}

Com os resultados das execuções, ficou clara que a instrução que funciona como barreira de memória não causa um efeito significativo na execução dos algoritmos, sendo notável somente o fato de que a sua presença garante que ambos os algoritmos possam viabilizar exclusão mútua com 100 threads, o que não se observa na ausência do (ss). 

O aumento na quantidade de threads, bem como o do tempo total provocaram aumentos no tempo total. Isso é facilmente explicável, pelo fato de que aumenta-se a quantidade de threads para acessar a SC no primeiro caso, e depois as tentativas de acesso. 

Outro fenômeno curioso é a uniformidade de resultados obtidos em $\mu_{bakery}$ e a quase uniformidade no caso de $\mu_{gate}$. Ainda que pareça especulativo, é notável que a relação entre tempo total e quantidades de threads é idêntica entre experimentos que apresentam média das médias com resultados rigorosa ou aproximadamente iguais. O motivo disso provavelmente deve-se ao fato de haver proporções idênticas de tempo e threads possibilitando resultados parecidos ou mesmo idênticos.

Há de observar-se também que a o desvio padrão diminuía no caso do bakery de acordo com o aumento das quantidades de tempo e threads, enquanto no algoritmo gate, observa-se o desvio padrão diminui juntamente à diminuição da razão tempo/threads.

Os resultados mais interessantes observam-se na tabela \ref{escalador}. Veja: \begin{itemize}
\item $\mu_{bakery}$ mantem-se imutável, enquanto $\mu_{gate}$ diminui com a presença de ss associada ao \texttt{RR} e \texttt{FIFO}
\item Levando em conta os desvios padrão: \begin{itemize}
\item Para as execuções de bakery, o valor médio vai decresendo de escalonador para escalonador.
\item Ainda levando o bakery em consideração, não parece ser possível traçar uma correlação entre os resultados com e sem ss
\item Para as execuções de gate, o único padrão observável é que a média dos resultados com ss é sempre maior que o sem.
\end{itemize}
\end{itemize}
\end{document}
