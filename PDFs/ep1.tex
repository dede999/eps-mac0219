%%
%% Author: André Luiz Abdalla 8030353
%%          Mauricio Abreu Cardoso 0000000
%% 19/05/18
%%

% Preamble
\documentclass[11pt]{article}
\usepackage{basic}
\author{André Luiz Abdalla $8030353$ \\ Mauricio Luiz Abreu Cardoso $6796479$}
\title{Relatório EP1}

% Document
\begin{document}
    \maketitle

    \section{Introdução}

    O algoritmo utilizado para a multiplicação das matrizes é o mais básico e simples para se resolver tal problema,
    dado que não queríamos inserir dificuldades desnecessárias

    \begin{lstlisting}[caption=Função para multiplicar matrizes]
double** multiplicaMatrizes(double **mat_a, double **mat_b, int la, int ca, int cb) {
    int c, d, k;
    double sum = 0;

    double **mat_c = malloc (ca * sizeof(double *));
    for (c = 0; c < la; c++)
        mat_c[c] = malloc (cb * sizeof(double));


    for (c = 0; c < la; c++) {
        for (d = 0; d < cb; d++) {
            for (k = 0; k < ca; k++) {
                sum = sum + mat_a[c][k]*mat_b[k][d];
            }
            mat_c[c][d] = sum;
            sum = 0;
        }
    }

    return mat_c;
}
    \end{lstlisting}

    Tendo a ferramenta multiplicadora e o que demais foi necessário, cuidamos de como paralelizar \begin{itemize}
        \item[\textbf{Pthreads}] [Como fizemos as pthreads]
        \item[\textbf{Open MP}] [Como fizemos com Open MP]
    \end{itemize}

\end{document}
