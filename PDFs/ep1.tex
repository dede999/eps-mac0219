%%
%% Author: André Luiz Abdalla 8030353
%%          Mauricio Abreu Cardoso 0000000
%% 19/05/18
%%

% Preamble
\documentclass[11pt]{article}
\usepackage{basic}
\author{André Luiz Abdalla $8030353$ \\ Mauricio Luiz Abreu Cardoso $6796479$}
\title{Relatório EP1}

% Document
\begin{document}
    \maketitle

    \section{Introdução}

    O algoritmo utilizado para a multiplicação das matrizes é o mais básico e simples para se resolver tal problema,
    dado que não queríamos inserir dificuldades desnecessárias

    \begin{lstlisting}[caption=Função para multiplicar matrizes]
double** multiplicaMatrizes(double **mat_a, double **mat_b, int la, int ca, int cb) {
    int c, d, k;
    double sum = 0;

    double **mat_c = malloc (ca * sizeof(double *));
    for (c = 0; c < la; c++)
        mat_c[c] = malloc (cb * sizeof(double));


    for (c = 0; c < la; c++) {
        for (d = 0; d < cb; d++) {
            for (k = 0; k < ca; k++) {
                sum = sum + mat_a[c][k]*mat_b[k][d];
            }
            mat_c[c][d] = sum;
            sum = 0;
        }
    }

    return mat_c;
}
    \end{lstlisting}

    Tendo a ferramenta multiplicadora e o que demais foi necessário, cuidamos de como paralelizar \begin{itemize}
        \item[\textbf{Pthreads}] A multiplicação de matrizes foi paralelizada de acordo com o número de núcleos disponíveis. Se a matriz A tiver mais linhas do que o processador tiver cores, serão alocadas threads em mesma quantidade que o número de cores, e as multiplocações das linhas da matriz A serão divididas entre as threads, de modo igual. Se o resto da divisão de linhas da matriz A pelo número de cores for diferente de 0, a última thread terá menos linhas a serem multiplicadas. Vale ressaltar que no caso que o número de linha de A é menor que o número de cores, a quantidade de threads será igual a quantidade de linhas, e cada thread receberá somente uma linha.
        \item[\textbf{Open MP}] Para o programa usando essa ferramenta, incluímos uma linha
        \footnote{\texttt{$\#$pragma omp parallel for schedule(static, 2) num\_threads(4) ordered}}. Além de garantir o
        paralelismo para o for mais externo, permite que determinemos quantas threads haverão (4 neste caso) e que ordenemos sua execução
    \end{itemize}

\end{document}
